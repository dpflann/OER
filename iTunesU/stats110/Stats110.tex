\documentclass[11pt, oneside]{article}   	% use "amsart" instead of "article" for AMSLaTeX format
\usepackage{geometry}                		% See geometry.pdf to learn the layout options. There are lots.
\usepackage{amsmath}				% use amsmath
\geometry{letterpaper}                   		% ... or a4paper or a5paper or ... 
%\geometry{landscape}                		% Activate for rotated page geometry
%\usepackage[parfill]{parskip}    		% Activate to begin paragraphs with an empty line rather than an indent
\usepackage{graphicx}				% Use pdf, png, jpg, or eps§ with pdflatex; use eps in DVI mode
								% TeX will automatically convert eps --> pdf in pdflatex		
\usepackage{amssymb}
\usepackage{color}
\usepackage{hyperref}				% Allow links between table of contents and sections
\hypersetup{
	colorlinks=true,
	linkcolor=black,
	urlcolor=red,
	linktoc=all
}

%SetFonts

%SetFonts


\title{Stats 110 Notes}
\author{Daniel Flannery}
%\date{}							% Activate to display a given date or no date

\begin{document}
\maketitle
\tableofcontents

% Lecture 1 notes
\section{Lecture 1: Probability and Counting}					
	\subsection{Notes}
	\begin{itemize}
		\item A \textbf{Sample Space} is the set of all possible outcomes of an experiment
		\item An \textbf{event} is a subset of the sample space.
		\item \textbf{Naive Definition of Probability}: \[P(A) = \frac{\# \hspace{2mm} of \hspace{2mm} favorable \hspace{2mm} outcomes}{\# \hspace{2mm} possible \hspace{2mm} outcomes}\]
		\item \textbf{The Multiplication Rule} - if you have experiment with $n$ possible outcomes, and for each outcome of the first experiment there 		are $n$ outcomes for the second experiments, ..., for each $r^{th}$ experiment there are $n_{r}$ outcomes, then there are $n_{1}*n_{2}*...*n_{r}$ overall possible outcomes.
		Assumes all outcomes are equally likely within a finite sample space
		\item \textbf{Binomial Coefficient}: \[ \binom{n}{k} = \frac{n!}{(n-k)!k!} \] $0$ if $k > n$\\
		\\
		How do you get there?
		Number of subsets of size $k$ of group $n$. Choosing $k$ from $n$ in order: $n*(n-1)(n-2)*...*(n-k+1)$
		Now if order does not matter: $\frac{n*(n-1)(n-2)*...*(n-k+1)}{k!}$
		Now through canceling we reach the formula for the binomial coefficient.
		\end{itemize}
	\subsection{Example: Full House, 5 Card Hand}
		\paragraph{}
			There are 13 suits. We must choose one suit then 3 of the 4 possible cards. Then we must choose another suit of the remaining 12, then select 2 of the 4 possible cards. Then divide by the sample space which is all possible 5 card hands from a deck of 52 cards.
		\begin{center}
			\[ \frac{13\binom{4}{3} * 12\binom{4}{2}}{\binom{52}{5}} \]
		\end{center}
	\subsection{Sampling Table}
	% Begin Table for fundamental 
	\begin{center}
		\begin{tabular}{ |c|c|c| }
	 	\hline
		\multicolumn{3}{|c|}{ \textbf{Sampling Table} } \\
 		\hline
 		&\textbf{Order Matters}	&\textbf{Order Doesn't Matter} \\
 		\hline
 		&&\\
		\textbf{Replace}	&$n^k$ 	&$\binom{n+k-1}{k}$\\
		&&\\
		\textbf{Don't Replace} 	&$n(n-1)(n-2)...(n-k+1)$ 	& $\binom{n}{k}$\\
		&&\\
 		\hline
		\end{tabular}
	\end{center}
	
\break
% Lecture 2 notes
\section{Lecture 2: Story Proofs, Axioms of Probability}
	\subsection{Notes}
	\begin{itemize}
		\item Classic and Fundamental Relationship: \[ \binom{n}{k} = \binom{n}{n-k} \]
		\item Exploring picking $k$ times from set of $n$ objects with replacement where order doesn't matter. Let's try out a few cases:
		\paragraph{Exploration}
		\begin{itemize}
			\item Case 1: $k = 0 \rightarrow \binom{n-1}{0} = 1$
			\item Case 2: $k = 1 \rightarrow \binom{n}{1} = n$
			\item Case 3 (Simplest non-trivial example): $n = 2 \rightarrow \binom{k+1}{k} = \binom{k+1}{1} = k + 1$
			\item This is equivalent to the number of ways to put $k$ indistinguishable particles into $n$ distinguishable boxes, which can be visualized as such: \[ \cdot \cdot \cdot \mid \mid \cdot \cdot \mid \cdot \]
			The $\mid$ represent partitions for the 'boxes' and the $\cdot$ is a particle. There are then $k$ $\cdot$'s and $n-1$ $\mid$'s. This can be simplified again to be thought of just arranging symbols, and you are choosing to place
			the symbols. There are $k + n - 1 = n + k -1$ symbols, so choose where to place the separators or the particles. Choose where the $\cdot$'s are, and the $\mid$'s are then positioned, and vice versa. \[ \binom{n + k - 1}{k} = \binom{n + k - 1}{n - 1} \]
		\end{itemize}
		\item \textbf{Story Proof}: proof by interpretation.
		\begin{itemize}
			\item \[ n \binom{n - 1}{k - 1} = k \binom{n}{k} \] Pick $k$ people out of $n$ with designated as the president, so you can first pick the president out of $n$, then select who can be in the club. Or you can pick the club $k$ out of $n$ then choose the president from the $k$ people.
			\item \textbf{Vandermonde Identity}: \[ \binom{m + n}{k} = \sum_{j = 0}^{k} \binom{m}{j} \binom{n}{k-j} \] You have two groups with labeled people containing $m$ and $n$ people. You need to select $k$ people total from both groups. The summation
			refers to the combinations of selecting $j$ from $m$ and $k-j$ from $n$ such that you have $j + k - j = k$ total people from the $m + n$ total people in two groups. Each instance is multiplied together due to the multiplication rule.
		\end{itemize}
		\item \textbf{The Non-Naive Definition of Probability}
			\begin{itemize}
				\item A probability sample consists of \textbf{S} and \textbf{P} where \textbf{S} is a sample space and \textbf{P}, a function which takes an event $A \subseteq S$ as input and returns $P(A) \in [0, 1]$ as output such that:
				\begin{itemize}
					\item (1) $P(\emptyset) = 0$, $P(S) = 1$
					\item (2) \[P( \cup_{n=1}^{\infty} A_{n} ) = \sum_{n=1}^{\infty} P(A_{n})\] if $A_1, A_2, ..., A_n$ are disjoint.
				\end{itemize}
			\end{itemize}
	\end{itemize}
% Lecture 3 notes
\section{Lecture 3: Birthday Problem, Properties Probability}
	\subsection{Notes}
	\begin{itemize}
		\item Birthday Problem: k people, find the probability that 2 have the same birthday. Exclude February 29, assume other 365 days are equally likely.
			If $k > 365$, the probability is 1 by the pigeonhole principle.
			If $k \leq 365$, the probability of no match:
				\[
					P(M^c)= \frac{365 \cdot 364 million\cdot 363 \cdots (365 - k + 1)}{365^k}
				\]
			The probability of a match is: $1 - P(M^c)$
				\[
					P(M) = 1 - P(M^c) = 1 - \frac{365 \cdot 364 \cdot 363 \cdots (365 - k + 1)}{365^k}
				\]
			Imagine pairing up the people. There are $\binom{k}{2}$ pairs.
		\item Axioms:
			\begin{enumerate}
				\item $P(\emptyset) = 0$, $P(S) = 1$
				\item $P( \cup_{n=1}^{\infty} = \sum_{n=1}{^\infty}P(A_n)$ if $A_i$ are disjoint.
			\end{enumerate}
		\item Properties:
			\begin{enumerate}
				\item $P(A^c) = 1 - P(A)$ Proof: $1 = P(S) = P(A \cup A^c) = P(A) + P(A^c)$ because $A \cap A^c = \emptyset$
				\item If $ A \subseteq B$, then $P(A) \leq P(B)$. Proof: $B = A \cup (B \cap A^c)$, disjoint, then $P(B) = P(A) + P(B \cap A^c) \geq P(A)$
				\item $P(A \cup B) = P(A) + P(B) - P(A \cap B)$. Proof: $P(A \cup B) = P(A \cup ( B \cap A^c ) = P(A) + P(B \cap A^c)$
					\[
					 	?= P(A) + P(B) - P(A \cap B), P(A \cap B) + P(B \cap A^c) = P(B)
					\]
					$A \cap B$ and $A^c \cap B$ are disjoint and the union is B
			\end{enumerate}
		\item Inclusion-Exclusion: Expanding upon most recent item. Example with 3 sets: A, B, C.
			\[
				P(A \cup B \cup C) = P(A) + P(B) + P(C) - P(A \cap B) - P(A \cap C) - P(B \cap C) + P(A \cap B \cap C)
			\]
			Generalized:
			\[
				P(A_1 \cup A_2 \cup \cdots \cup A_n)
			\]
			\[
				= \sum{j=1}^{n}P(A_j) - \sum{i < j}P(A_i \cap A_j) + \sum_{i < j < k}P(A_i \cap A_j \cap A_k) + \cdots + (-1)^{n+1}P(A_1 \cap \cdots \cap A_n)
			\]
		\item de Montmort's Matching Problem. \href{http://www.math.uah.edu/stat/urn/Matching.html}{link}
			What is the probability that the $n^{th}$ card has the label $n$?
			\[
				= 1 - 1/e
			\]
	\end{itemize}
\end{document}  