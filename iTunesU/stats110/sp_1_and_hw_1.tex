\documentclass[11pt, oneside]{article}   	% use "amsart" instead of "article" for AMSLaTeX format
\usepackage{geometry}                		% See geometry.pdf to learn the layout options. There are lots.
\usepackage{amsmath}				% use amsmath
\geometry{letterpaper}                   		% ... or a4paper or a5paper or ... 
%\geometry{landscape}                		% Activate for rotated page geometry
%\usepackage[parfill]{parskip}    		% Activate to begin paragraphs with an empty line rather than an indent
\usepackage{graphicx}				% Use pdf, png, jpg, or eps§ with pdflatex; use eps in DVI mode
								% TeX will automatically convert eps --> pdf in pdflatex		
\usepackage{amssymb}

%SetFonts

%SetFonts


\title{Stats 110 Strategic Practice and Homework 1}
\author{Daniel Flannery}
%\date{}							% Activate to display a given date or no date

\begin{document}
\maketitle
% Strategic Practice
\section{Strategic Practice}
\subsection{Naive Definition of Probability}
	\begin{enumerate}
		\item 
			\begin{enumerate}
			\item The probability of rolling 21  with 4 fair dice is $>$ The probability of rolling 22 with for 4 fair dice.\\\\
			4 fair dice have a sum in the range $[4, 24]$ with $6^4 = 1296$ combinations.
			A 21 can be constructed from the partitioned into $(6, 6, 6, 3), (6, 6, 5, 4), (6, 5, 5, 5)$; while 22 can partitioned into $(6, 6, 6, 4), (6, 6, 5, 5)$
			For example, for $(6, 6, 6, 3)$ there are $4$ possibilities: 4 dice, choose one to be 3, the rest are 6.
			For example, for $(6, 6, 5, 4)$ there are $\frac{4!}{2!} = 12$ possibilities because there are 4 dice, you must choose one to be 5, one to be 4, and the rest to be 6. Divide by $2!$ to account for the possible orderings of 6 due to repetition.
			For example, for $(6, 6, 5, 5)$ there are $\frac{4!}{2! 2!} = 6$ possibilities because there are two possible values and order does not matter.
			Therefore, there are $4 + 4 + 12 = 20$ ways to get 22.
			Therefore, there are $4 + 6 = 10$ ways to get 21.
			\item The probability of a random 2 letter word being a palindrome $=$ The probability of a random 3 letter word being a palindrome.\\\\
			For a two letter random word, there are $26^2$ outcomes. A two letter word is a palindrome when both letters are the same, so there are 26 palindrome and the probability is $\frac{26}{26^2}$.
			For a three letter random word, there are $26^3$ outcomes. A three letter word is a palindrome when the first and third letter are the same, so there are 26 letters, then for each combination, the second letter can be any of the 26 letters.
			Then by the multiplication rule, there are $26 * 26$ palindrome and the probability is $\frac{26^2}{26^3}$
			\[ P(2 \hspace{2mm} letter \hspace{2mm} palindrome) = P(3 \hspace{2mm} letter \hspace{2mm} palindrome) = \frac{26}{26^2} = \frac{26^2}{26^3} \]
			A three letter palindrome is an expansion of a two letter palindrome.
			\end{enumerate}
		\item Random 5 card poker hand from a standard deck
			\begin{enumerate}
				\item A flush (all 5 cards of same suit, excluding royal flush) \\\\
					There are 4 suits. For each suit there are 13 cards. There $\binom{13}{5}$ ways to choose a 5 card hand for each suit. 1 hand is excluded, the royal flush.\\
					The probability is: \[ \frac{(\# suits * ( allowable \hspace{2mm} hands )}{all \hspace{2mm} possible \hspace{2mm} hands} \]
					\[ \frac{4 (\binom{13}{5} - 1)}{\binom{52}{5}} \]
				\item Two pair \\\\
					Choose 2 ranks of 13 possible. Choose 2 of the 4 possible cards for each rank. Choose 1 of the remaining cards that is not of the 2 selected ranks. Divide by all possible hands.
					\[ \frac{\binom{13}{2} \cdot \binom{4}{2} \cdot \binom{4}{2} \cdot 44}{\binom{52}{5}} \]
			\end{enumerate}
		\item Paths on a grid from $(0, 0)$ to $(110, 111$) where the only moves are one unit up or one unit down.
			\begin{enumerate}
				\item Encode the moves as such: \textit{U} for up and \textit{R} for right. There are $110 - 0$ \textit{U}'s and $111 - 0$ \textit{R}'s. Therefore the set of moves can be written as: \textit{URURURRRR...UR}. There are then $110 + 111 = 221$ symbols to be
				positioned. To determine the number of paths choose a symbol to position: \[ \binom{221}{110} = \binom{221}{111} \]
				\item Paths from $(0, 0)$ to $(210, 211)$ that go through $(110, 111)$ \\\\
				From above there are $\binom{221}{110}$ paths from $(0, 0)$ to $(110, 111)$. Now merely need to find the number of paths from $(110, 111)$ to $(210, 211)$. This is then $210 - 110 = 100$ more \textit{R}'s and $211 - 111 = 100$ more \textit{U}'s.
				Therefore $\binom{200}{100}$ paths. By the multiplication rule then there are \[ \binom{221}{110} \cdot \binom{200}{100} \] total paths.
			\end{enumerate}
		\item A \textit{no-repeat-word} is a sequence of 26 letters unique and with no repetition. Show that the probability that a \textit{no-repeat-word} using all 26 letters selected at random with all \textit{no-repeat-word}  equally likely is very close to $1/e$. \\\\
			There are $26!$ total \textit{no-repeat-words} of length 26. To create a \textit{no-repeat-word} of length $k$, first select $k$ letters from the 26. $\binom{26}{k}$. Then those letters can be arranged into $k!$ words. Therefore there are $\binom{26}{k}k!$
			\textit{no-repeat-words} with $k$ letters. To count all possible \textit{no-repeat-words}, simply sum all the possible words of each possible length: $[1, 26]$. Then, the probability of a \textit{no-repeat-word} having all 26 letters is:
			\[ 
				= \frac{26!}{\sum_{k=1}^{26} \binom{26}{k} k!} = \frac{26!}{\sum_{k=1}{26} \frac{26!}{k!(26-k)} k!}
			\]
			\[
				= \frac{26!}{26! \cdot \sum_{k=1}^{26} \frac{k!}{k!(26-k)!} } = \frac{1}{\sum_{k=1}^{26} \frac{1}{(26-k)!}}
			\]
			\[
				= \frac{1}{ \frac{1}{25!} + \frac{1}{24!} + \cdots + \frac{1}{1!} + 1} = 1 + \frac{1}{1!} + \frac{1}{2!} + \cdots + \frac{1}{25!}
			\]
			which is equivalent to the Taylor series $e^x = 1 + x + \frac{x^2}{2!} + \cdots + \frac{x^{25}}{25!}$ evaluated at $x=1$.
	\end{enumerate}

\subsection{Story Proofs}

% Homework
\pagebreak
\section{Homework}

\end{document}  