\documentclass[11pt, oneside]{article}   	% use "amsart" instead of "article" for AMSLaTeX format
\usepackage{geometry}                		% See geometry.pdf to learn the layout options. There are lots.
\usepackage{amsmath}				% use amsmath
\geometry{letterpaper}                   		% ... or a4paper or a5paper or ... 
%\geometry{landscape}                		% Activate for rotated page geometry
%\usepackage[parfill]{parskip}    		% Activate to begin paragraphs with an empty line rather than an indent
\usepackage{graphicx}				% Use pdf, png, jpg, or eps§ with pdflatex; use eps in DVI mode
								% TeX will automatically convert eps --> pdf in pdflatex		
\usepackage{amssymb}
\usepackage{color}
\usepackage{hyperref}				% Allow links between table of contents and sections
\hypersetup{
	colorlinks=true,
	linkcolor=black,
	urlcolor=red,
	linktoc=all
}
\usepackage[makeroom]{cancel}
\usepackage{qtree}

%SetFonts

%SetFonts


\title{Stats 110 Strategic Practice and Homework 2}
\author{Daniel Flannery}
%\date{}							% Activate to display a given date or no date

\begin{document}

\maketitle
\tableofcontents

% Strategic Practice
\section{Strategic Practice}
\subsection{Inclusion-Exclusion}	
	\begin{enumerate}
		\item For 7 people, what is the probability that all 4 season occur at least once among their birthdays assuming all seasons are equally likely.
			$A_i$ is the probability that there are no birthdays in the $i^{th}$ season. Probability that all seasons occur at least once is $1 - P(A_1 \cup A_2 \cup A_3 \cup A_4)$.
			$A_1 \cap A_2 \cap A_3 \cap A_4 = \emptyset$
			\[
				P(A_1 \cup A_2 \cup A_3 \cup A_4) = \sum_{i}^{4}P(A_i) - \sum_{i=1}^{3}\sum_{j > i}P(A_i \cap A_j) + \sum_{i=1}^{3}\sum_{j>i}\sum_{k>j}P(A_i \cap A_j \cap A_k)
			\]
			\[
				= 4P(A_1) - 6P(A_1 \cap A_2) + 4P(A_1 \cap A_2 \cap A_3)
			\]
			$P(A_1) = (3 / 4)^{7}$,
			$P(A_1 \cap A_2) = \frac{1}{2^{7}}$,
			$P(A_1 \cap A_2 \cap A_3) = \frac{1}{4^{7}}$
			\[
				P(A_1 \cup A_2 \cup A_3 \cup A_4)  = 4(\frac{3}{4^{7}}) - 6(\frac{1}{2^{7}}) + 4(\frac{1}{4^{7}})
			\]
			\[
				\rightarrow 1 - [4(\frac{3}{4^{7}}) - 6(\frac{1}{2^{7}}) + 4(\frac{1}{4^{7}})]
			\]
			The probability that there are no birthdays in all the season is the $\emptyset$. The probability that there are no birthdays in a given season means you have reduced the number of season by 1,
			there are now 3 seasons to choose from. And so on for 2 seasons not occurring, and so on for 3 seasons not occurring. Each time you are making the event less probable by shrinking the number of options.
		\item Picking Classes Randomly
			\begin{enumerate}
				\item Naive Method: There are 7 classes to take total, there are 30 total classes, 6 each day of the week. 7 classes can be taken either with 2 days with 2 classes and 3 days with 1 or 1 day with 3 classes and 4 days with 1:
					(2, 2, 1, 1, 1) or (3, 1, 1, 1, 1). First select the days of the week that will have more than one course, then for those days select the required number out of the 6 possible classes, then for the remaining days, select
					1 of the possible 6 classes.
					\[
						\frac{\binom{5}{2}\binom{6}{2}\binom{6}{2}\binom{6}{1}\binom{6}{1}\binom{6}{1} + \binom{5}{1}\binom{6}{3}\binom{6}{1}\binom{6}{1}\binom{6}{1}\binom{6}{1}}{\binom{30}{7}}
					\]
				\item Inclusion-Exclusion Method: To find the probability of classes on each day of the week, consider the complement, that is the probability that at least one day does not have classes.
					To take 7 classes total, there must be at least 2 days with classes because of the constraint that each day has 6 classes. Let $B_i = A_{i}^{c}$, where $A_i$ is the probability of class on that day.
					\[
						P(atLeastOneDayWithNoClass) = 
					\]
					\[
						\sum_{i}P(NoClassOnDay A_i)
					\]
					\[
						- \sum_{i<j}P(NoClassOnDay A_i and A_j)
					\]
					\[
						+ \sum_{i<j<k}P(NoClassOnDay A_i and A_j and A_k)
					\]
					There must be two days with classes, therefore there is no need to consider no classes on 4 days and no classes on 5 days.
					\[
						P(B_1 \cup B_2 \cup \cdots \cup B_5) = \sum_{i}P(B_i) - \sum_{i<j}P(B_i \cap B_j) + \sum_{i<j<k}P(B_i \cap B_j \cap Back_k)
					\]
					Now, to fill in the probabilities:
					\[
						P(B_1) = \frac{\binom{24}{7}}{\binom{30}{7}}, P(B_1 \cap B_2) = \frac{\binom{18}{7}}{\binom{30}{7}}, P(B_1 \cap B_2 \cap B_3) = \frac{\binom{12}{7}}{\binom{30}{7}}
					\]
					Other intersections are similar.
					\[
						P(B_1 \cup B_2 \cup \cdots \cup B_5)  = 5 \frac{\binom{24}{7}}{\binom{30}{7}} - \binom{5}{2} \frac{\binom{18}{7}}{\binom{30}{7}} + \binom{5}{3} \frac{\binom{12}{7}}{\binom{30}{7}}
					\]
					Therefore:
					\[
						P(A_1 \cup A_2 \cup A_3 \cup A_4 \cup A_5) = 1 - P(B_1 \cup B_2 \cup \cdots \cup B_5)  = 5 \frac{\binom{24}{7}}{\binom{30}{7}} - \binom{5}{2} \frac{\binom{18}{7}}{\binom{30}{7}} + \binom{5}{3} \frac{\binom{12}{7}}{\binom{30}{7}}
					\]
			\end{enumerate}
	\end{enumerate}
\subsection{Independence}
% Homework
\pagebreak
\section{Homework}
\end{document} 